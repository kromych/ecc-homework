\documentclass[12pt]{article}
\usepackage[utf8]{inputenc}
\usepackage{amsmath, amssymb, amsthm}
\usepackage{graphicx}
\usepackage[margin=1in]{geometry}
\usepackage{hyperref}
\usepackage{mathrsfs}
\usepackage{tcolorbox}

\title{ECC, problem 01.**08}
\author{Roman}
\date{\today}

\begin{document}

\maketitle

\section*{Problem}
Prove that three points on an elliptic curve ${E}$ over the set of rational numbers $\mathbb{Q}$ are collinear
iff they add to the identity element $\mathcal{O}$. To simplify the proof, you may assume that the three points
are distinct.

\pagebreak

\section*{Solution}
\begin{proof}


\item{\textbf{Sufficient condition.}}


Let $P, Q, R$ be three points on $E$ such that $P + Q + R = \mathcal{O}$. We want to show that $P, Q, R$ are collinear, i.e.

\begin{tcolorbox}[colframe=blue, colback=gray!10]
\begin{equation*}
    P + Q + R = \mathcal{O} \implies P, Q, R \text{ are  collinear}
\end{equation*}
\end{tcolorbox}

The $x$-coordinates of $P + Q + R$ is (from the explicit formula for the addition law on ${E}$)
\begin{equation*}
    x_{P+Q+R} = \frac{(y_{P+Q}-y_R)^2}{(x_{P+Q}-x_R)^2} - x_{P+Q} - x_R
\end{equation*}

As $P + Q + R = \mathcal{O}$, we have $x_{P+Q} = x_R$. From that, using the explicit formula for the addition law on ${E}$
for the second time,
\begin{equation*}
    x_{P+Q} = x_R = \frac{(y_P-y_Q)^2}{(x_P-x_Q)^2} - x_P -x_Q
\end{equation*}
or
\begin{equation*}
    \frac{(y_P-y_Q)^2}{(x_P-x_Q)^2} = x_P + x_Q + x_R
\end{equation*}

The group law on ${E}$ is commutative and associative, so from $P + Q + R = \mathcal{O}$ with choosing another
order of summation one gets:
\begin{equation*}
    \frac{(y_P-y_Q)^2}{(x_P-x_Q)^2} = \frac{(y_P-y_R)^2}{(x_P-x_R)^2} = \frac{(y_Q-y_R)^2}{(x_Q-x_R)^2}
\end{equation*}
which states that the slopes of the lines $PQ$, $PR$, $QR$ are equal. Therefore, $P, Q, R$ are collinear.


\item{\textbf{Necessary condition.}}


Let $P, Q, R$ be three points on ${E}$ such that $P, Q, R$ are collinear. We want to show that $P + Q + R = \mathcal{O}$, i.e.

\begin{tcolorbox}[colframe=blue, colback=gray!10]
    \begin{equation*}
        P + Q + R = \mathcal{O} \impliedby P, Q, R \text{ are  collinear}
    \end{equation*}
\end{tcolorbox}

From the associative property of the group law on ${E}$, we have
\begin{equation*}
    (P + Q) + R = P + (Q + R)
\end{equation*}
For the $x$-coordinates of the both sides of the equation above, we have
\begin{equation*}
    x_{(P + Q) + R} = \frac{(y_{P+Q}-y_R)^2}{(x_{P+Q}-x_R)^2} - x_{P+Q} - x_R
\end{equation*}
and
\begin{equation*}
    x_{P + (Q + R)} = \frac{(y_P-y_{Q+R})^2}{(x_P-x_{Q+R})^2} - x_P - x_{Q+R}
\end{equation*}
which combined together give
\begin{equation*}
    \frac{(y_{P+Q}-y_R)^2}{(x_{P+Q}-x_R)^2} - x_{P+Q} - x_R = \frac{(y_P-y_{Q+R})^2}{(x_P-x_{Q+R})^2} - x_P - x_{Q+R}   
\end{equation*}

Using the explicit formula for the addition law on ${E}$, we have for the $x$-coordinates of $P + Q$ and $Q + R$:
\begin{equation*}
    x_{P+Q} = \frac{(y_P-y_Q)^2}{(x_P-x_Q)^2} - x_P - x_Q
\end{equation*}
and 
\begin{equation*}
    x_{Q+R} = \frac{(y_Q-y_R)^2}{(x_Q-x_R)^2} - x_Q - x_R
\end{equation*}
Substituting these into the equation above, it folows that the following equation holds for any three collinear points $P, Q, R$:
\begin{equation*}
    \frac{(y_{P+Q}-y_R)^2}{(x_{P+Q}-x_R)^2} - \frac{(y_P-y_{Q+R})^2}{(x_P-x_{Q+R})^2} = 0   
\end{equation*}
as the slopes of the lines $PQ$ and $QR$ are equal. This equation can be rewritten as
\begin{equation*}
    \frac{(y_{P+Q}-y_R)^2}{(x_{P+Q}-x_R)^2} = \frac{(y_P-y_{Q+R})^2}{(x_P-x_{Q+R})^2}
\end{equation*}
For that to hold for any three collinear points $P, Q, R$, the following equation must hold for any two points $P, Q$:
\begin{equation*}
    x_{P+Q} - x_R = x_P - x_{Q+R} = 0
\end{equation*}
That proves that $P + Q + R = \mathcal{O}$.


\item{\textbf{Conclusion.}}


We have shown that three points on an elliptic curve ${E}$ over the set of rational numbers $\mathbb{Q}$ are collinear
iff they add to the identity element $\mathcal{O}$:

\begin{tcolorbox}[colframe=blue, colback=gray!10]
    \begin{equation*}
        P + Q + R = \mathcal{O} \iff P, Q, R \text{ are  collinear}
    \end{equation*}
\end{tcolorbox}


\end{proof}

\end{document}
