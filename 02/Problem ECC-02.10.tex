\documentclass[12pt]{article}
\usepackage[utf8]{inputenc}
\usepackage{amsmath, amssymb, amsthm}
\usepackage{graphicx}
\usepackage[margin=1in]{geometry}
\usepackage[colorlinks=true, linkcolor=blue, citecolor=green, filecolor=magenta, urlcolor=red]{hyperref}
\usepackage{mathrsfs}
\usepackage{tcolorbox}
\usepackage{empheq}

\title{ECC, problem 02.$^\dagger$10}
\author{Roman}
\date{\today}

\begin{document}

\maketitle

\section*{Problem}

Let $E/\mathbb{Q}$ be an elliptic curve. Prove that $E[m]$ has $m^2$ points of order $m$.

\pagebreak

\section*{Solution}
\begin{proof}

As follows from the Mordell-Weil theorem,
\begin{equation}
    E[m] \cong Z/mZ \times Z/mZ
\end{equation}
Employing that fact makes it now obvious that
\begin{equation}
    \#E[m] = m^2
\end{equation}

\end{proof}

Another way to see this might be counting the roots of the $m$-division polynomial $\psi_m(x)$.


For odd $m$, the division polynomial is of power $(m^2-1)/2$ so it has that many roots. Adding points with $-y$ and $\infty$ gives 
\begin{equation}
    \#E[m] = \frac{m^2-1}{2} * 2 + 1 = m^2
\end{equation}
i.e. $m^2$ points.

For even $m$, the division polynomial is a product of a polynomial of power $(m^2-4)/2$ and $y(x)$ ($y$ is the $y$-coordinate of the point on the curve).
The first factor gives $(m^2-4)/2$ roots. Adding to that points symmetric across $x$-axis, then adding to that $3$ points where $y(x) = 0$ and $\infty$
\begin{equation}
    \#E[m] = \frac{m^2-4}{2} * 2 + 3 + 1 = m^2
\end{equation}
i.e. $m^2$ points. The gap in this treatment is the repeated roots of the division polynomials. I'm not sure how to deal with that.

\end{document}
