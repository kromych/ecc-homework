\documentclass[12pt]{article}
\usepackage[utf8]{inputenc}
\usepackage{amsmath, amssymb, amsthm}
\usepackage{graphicx}
\usepackage[margin=1in]{geometry}
\usepackage[colorlinks=true, linkcolor=blue, citecolor=green, filecolor=magenta, urlcolor=red]{hyperref}
\usepackage{mathrsfs}
\usepackage{tcolorbox}
\usepackage{empheq}

\title{ECC, problem 02.***08}
\author{Roman}
\date{\today}

\begin{document}

\maketitle

\section*{Problem}

Let $E(\mathbb{Q}) : y^2 = f(x) = x^3 + Ax + B$. Prove that
\begin{equation*}
    \frac{d^2 y}{dx^2} = \frac{2 f''(x)f(x) - f'(x)^2}{4 y f(x)} = \frac{\psi_3(x)}{4 y f(x)}
\end{equation*}
where $\psi_3(x) = 3x^4 + 6Ax^2 + 12Bx - A^2$ is the \textbf{third division polynomial} of
$E$.\footnote{
    It is a polynomial whose roots are the $x$-coordinates of the 3-torsion points of $E$.
    In general, there is an $m$-th division polynomial $\psi_m(x)$ whose roots give the $x$-coordinates
    of the $m$-torsion points of $E$. See \href{https://en.wikipedia.org/wiki/Division_polynomials}{this link} for information.
}


Use this to deduce that a point $P = (x,y) \in E $ (not equal to $\infty$)
is a point of order three if and only if $P \neq \infty$ and $P$ is a point of inflection on the curve $E$.

\pagebreak

\section*{Solution}
\begin{proof}

First, we compute the second derivative of $y$ with respect to $x$. For the first derivative, we have
\begin{equation}
    2yy' = f'(x)
    \label{eq:1st-derivative}
\end{equation}
Differentiating both sides with respect to $x$ gives
\begin{equation}
    2y'^2 + 2yy'' = f''(x) 
\end{equation}
Solving equation (\ref{eq:1st-derivative}) for $y'$ and substituting into the above equation gives
\begin{equation}
    2y \left( \frac{f'(x)}{2y} \right)^2 + 2yy'' = f''(x)
\end{equation}
Now we can simplify the left-hand side to get the desired result:
\begin{equation}
    \frac{d^2 y}{dx^2} = \frac{2 f''(x)f(x) - f'(x)^2}{4 y f(x)}
\end{equation}
That holds for any implicit function $y^2 = f(x)$. For our curve $E(\mathbb{Q})$ that is
\begin{equation}
    y^2 = x^3 + Ax + B  
\end{equation}
we have
\begin{equation}
    \frac{d^2 y}{dx^2} = \frac{2 f''(x)f(x) - f'(x)^2}{4 y f(x)} \stackrel{\text{[some laborious algebra]}}{=} \frac{3x^4 + 6Ax^2 + 12Bx - A^2}{4 y f(x)} = \frac{\psi_3(x)}{4 y f(x)}
\end{equation}

To find the point of inflection, we need the roots of the third division polynomial $\psi_3(x)$. With a help of a computer, we find that \footnote{
    A curious property of the roots is $\sum_i x_i^2 = -4A$.  Interesting, like a norm of some vector (it's $\le 0$ due to the complex roots)!
    Computed $\sum_i x_i^2$ for $\psi_2$: $-2A$,  $\psi_4$ : $-10A$, $\psi_5$ : $-124/5A$ (can multiply by $m$?), $\psi_6$ : $-50A$, $\psi_7$ : $-88A$,
    $\psi_8$ : $-148A$. That last one is a polynomial of the 33rd power and takes 266 KiB, sure I was leaning on Sage Math for getting the explicit form
    for $\psi_m$ and Wolfram Mathematica for manipulating with roots. Can't compute more it appears. The question is what is that sequence, known or
    even useful perhaps?
}
\begin{align}
    x_1 = \frac{ \sqrt{\sqrt[3]{8 A^3+54 B^2}-2 A}-\sqrt{-\sqrt[3]{8 A^3+54 B^2}-\frac{6 \sqrt{6} B}{\sqrt{\sqrt[3]{8 A^3+54 B^2}-2 A}}-4 A}}{\sqrt{6}} \\
    x_2 = \frac{ \sqrt{\sqrt[3]{8 A^3+54 B^2}-2 A}+\sqrt{-\sqrt[3]{8 A^3+54 B^2}-\frac{6 \sqrt{6} B}{\sqrt{\sqrt[3]{8 A^3+54 B^2}-2 A}}-4 A}}{\sqrt{6}} \\
    x_3 = \frac{-\sqrt{\sqrt[3]{8 A^3+54 B^2}-2 A}-\sqrt{-\sqrt[3]{8 A^3+54 B^2}+\frac{6 \sqrt{6} B}{\sqrt{\sqrt[3]{8 A^3+54 B^2}-2 A}}-4 A}}{\sqrt{6}} \\
    x_4 = \frac{-\sqrt{\sqrt[3]{8 A^3+54 B^2}-2 A}+\sqrt{-\sqrt[3]{8 A^3+54 B^2}+\frac{6 \sqrt{6} B}{\sqrt{\sqrt[3]{8 A^3+54 B^2}-2 A}}-4 A}}{\sqrt{6}}
\end{align}
Using the discriminant $\Delta = 4A^3 + 27B^2$ of the cubic polynomial $f(x)$, we can simplify the above expressions to
\begin{align}
    x_1 = \frac{ \sqrt{\sqrt[3]{2 \Delta}-2 A}-\sqrt{-\sqrt[3]{2 \Delta}-\frac{6 \sqrt{6} B}{\sqrt{\sqrt[3]{2 \Delta}-2 A}}-4 A}}{\sqrt{6}} \\
    x_2 = \frac{ \sqrt{\sqrt[3]{2 \Delta}-2 A}+\sqrt{-\sqrt[3]{2 \Delta}-\frac{6 \sqrt{6} B}{\sqrt{\sqrt[3]{2 \Delta}-2 A}}-4 A}}{\sqrt{6}} \\
    x_3 = \frac{-\sqrt{\sqrt[3]{2 \Delta}-2 A}-\sqrt{-\sqrt[3]{2 \Delta}+\frac{6 \sqrt{6} B}{\sqrt{\sqrt[3]{2 \Delta}-2 A}}-4 A}}{\sqrt{6}} \\
    x_4 = \frac{-\sqrt{\sqrt[3]{2 \Delta}-2 A}+\sqrt{-\sqrt[3]{2 \Delta}+\frac{6 \sqrt{6} B}{\sqrt{\sqrt[3]{2 \Delta}-2 A}}-4 A}}{\sqrt{6}}
\end{align}

Introducing
\begin{align}
    M = \sqrt[3]{2 \Delta} - 2A \\
    N = \frac{6 \sqrt{6} B}{\sqrt{M}} - 6A
\end{align}
we further simplify the expressions for the roots to
\begin{align}
    x_1 = \frac{ \sqrt{M}-\sqrt{-M-N}}{\sqrt{6}} \\
    x_2 = \frac{ \sqrt{M}+\sqrt{-M-N}}{\sqrt{6}} \\
    x_3 = \frac{-\sqrt{M}-\sqrt{-M+N}}{\sqrt{6}} \\
    x_4 = \frac{-\sqrt{M}+\sqrt{-M+N}}{\sqrt{6}}
\end{align}

In the case of $B = 0$, the roots are:
\begin{align}
    x_1 = -\sqrt{\frac{2}{\sqrt{3}}+1} \quad \sqrt{-A}    \\
    x_2 = \sqrt{\frac{2}{\sqrt{3}}+1}  \quad \sqrt{-A}    \\
    x_3 = -\sqrt{\frac{2}{\sqrt{3}}-1} \quad \sqrt{A}     \\
    x_4 = \sqrt{\frac{2}{\sqrt{3}}-1}  \quad \sqrt{A} 
\end{align}

In the case of $A = 0$, the roots are:
\begin{align}
    x_1 = 0 \\
    x_2,x_3,x_4 = -\sqrt[3]{4 B}
\end{align}
    

\item{\textbf{Sufficient condition.}}


\item{\textbf{Necessary condition.}}


\item{\textbf{Conclusion.}}


\end{proof}


\end{document}